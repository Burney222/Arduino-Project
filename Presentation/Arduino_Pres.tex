\documentclass[xcolor=dvipsnames,10pt]{beamer}
%\documentclass[xcolor=dvipsnames,10pt]{beamer}

\usepackage{textpos} %for textblock

\usepackage[utf8]{inputenc}
\usepackage[ngerman]{babel}
\usepackage{listings}
\usepackage{slashed,subfigure}
\usepackage{xcolor}
\usefonttheme{serif}

%SI-UnitX
\usepackage[separate-uncertainty=true]{siunitx}
%\sisetup{separate-uncertainties=true}
\sisetup{
	per-mode = fraction, 			% Bruchstriche nutzen
	output-decimal-marker = {.}, 		% Setzt das Dezimaltrennzeichen als Punkt
	multi-part-units = brackets,
	exponent-product = \cdot,
}



\usepackage{graphicx, dsfont}
\usepackage{multimedia}
\usepackage{listings}

\newcommand{\Lagr}{\mathcal{L}}
\newcommand{\F}{\mathcal{F}}
\newcommand{\B}{\mathcal{B}}

\newcommand{\dd}{\mathrm{d}}

\newcommand{\ii}{\mathrm{i}}
\newcommand{\bos}{\boldsymbol}

\newcommand{\Nc}{N_{\text{c}}}
\newcommand{\Nf}{N_{\text{f}}}

% Defining often used decays
\newcommand{\nonres}{B_s \rightarrow f_2' \mu^+\!\mu^-}
\newcommand{\res}{B_s \rightarrow f_2' J\!/\!\psi}
\newcommand{\resmumu}{B_s \rightarrow f_2' J\!/\!\psi (\rightarrow \mu^+ \! \mu^-)}
\newcommand{\Jpsimumu}{J\!/\!\psi \rightarrow \mu^+ \! \mu^-}

%\usepackage[ngerman]{babel}  %Deutsche Bezeichnungen, Worttrennung
%\usepackage[utf8]{inputenc}  %Umlaute
%\usepackage{verbatim}
%\usepackage[latin1]{inputenc}  %Umlaute
%\usepackage{tikz}               %Für Kreise und co

\usecolortheme[named=Green]{structure}  %color scheme
\usetheme{Madrid}  %or Madrid, Boadilla, Warsaw
\setbeamertemplate{items}[square]  %ball, square, circle
%\setbeamertemplate{blocks}[default][shadow=true]  %default vs rounded
\setbeamertemplate{navigation symbols}{}  %no navigation symbols
\setbeamertemplate{footline}
{
\leavevmode
\hbox{%
\begin{beamercolorbox}[wd=.333333\paperwidth,ht=2.25ex,dp=1ex,center]{author in head/foot}
\usebeamerfont{author in head/foot}\insertshortauthor%~~(\insertshortinstitute)
\end{beamercolorbox}%
\begin{beamercolorbox}[wd=.333333\paperwidth,ht=2.25ex,dp=1ex,center]{title in head/foot}
\usebeamerfont{title in head/foot}\insertshorttitle
\end{beamercolorbox}%
\begin{beamercolorbox}[wd=.333333\paperwidth,ht=2.25ex,dp=1ex,right]{date in head/foot}
\usebeamerfont{date in head/foot}\insertshortdate{}\hspace*{2em}
%\insertframenumber{} / \inserttotalframenumber\hspace*{2ex}
\insertframenumber{} / \inserttotalframenumber \hspace*{2ex}
\end{beamercolorbox}}%
\vskip0pt
}

\newcommand{\sh}[1]{#1\hskip-5pt /}

\def\red{\color[rgb]{1,0,0}}
\def\green{\color[rgb]{0,1,0}}
\def\blue{\color[rgb]{0,0,1}}
\def\gray{\color[rgb]{0.45,0.45,0.45}}
\def\lgreen{\color[rgb]{0,0.5,0.5}}



%-----------------------------------
% Title page -----------------------
%-----------------------------------
\title[Arduino-Projekt]{\texorpdfstring{Arduino-Projekt: Gedächtnis-Training}{}} %-- \\ Managing  \\ using the ``Scrum'' framework
\author{Daniel Berninghoff}

\date[27. August 2015]{27. August 2015}



%\AtBeginSection[]{
%\frame<beamer>[noframenumbering]{ 
%\frametitle{Contents}   
%\tableofcontents[currentsection] 
% }
%  }


%\AtBeginSection[]
%{
 %  \begin{frame}[t]{Contents}
  %     \tableofcontents[currentsection,subsections]
  % \end{frame}
%}

\begin{document}

%---------------------------------------
% Titlepage ----------------------------
%---------------------------------------
\begin{frame}[noframenumbering]
\titlepage


% Two logos bottom-right and bottom-left on the Titlepage
\begin{figure}		% Bilder nebeneinander
	\includegraphics[scale=0.1]{Graphics/tudo_logo}
	%\caption{Gemeinsame Bildunterschrift} % optional
\end{figure}



\end{frame}


%---------------------------------------
% Table of Contents ----------------------------
%---------------------------------------
%\begin{frame}[t, noframenumbering]{Contents}
%\tableofcontents
%\vfill
%{\gray Based on arXiv:1003.5012}
%\end{frame}


%\section{Project: Analysis of new $\Lambda_b$ decays in a team}

%\frame<beamer>[noframenumbering]{ 
%\frametitle{Contents}   
%\tableofcontents[currentsection] 
% }

%\subsection{Motivation}



%--------------------------------------------
% Übersicht ---------------------------------
%--------------------------------------------
\begin{frame}\frametitle{Übersicht}
	\begin{columns}[c]
		\column{0.5\textwidth}
		\begin{enumerate} \itemsep3ex
			\item Idee
			\item Aufbau
			\item Ablauf
			\item Schwierigkeiten
			\item Ausblick
		\end{enumerate}
		
		\column{0.7\textwidth}
		\includegraphics[width=0.6\textwidth]{aufbau_gross.jpg}\\
		 {\small \textbf{Abbildung:} Aufbau}
	\end{columns}


\end{frame}






%--------------------------------------------
% Idee --------------------------------------
%--------------------------------------------
\begin{frame}\frametitle{Idee}
	\begin{itemize} \itemsep3ex
		\item Generierung und Ausgabe einer zufälligen Blink- und Ton-Sequenz 
		\item Länge der Startsequenz einstellbar
		\item Versuch, Sequenz durch entsprechende Tastendrücke zu wiederholen
		\item Bei erfolgreicher Eingabe: Erweiterung der Sequenz um eine LED bzw. einen Ton
		\item Speicherung der längsten Sequenz pro Session ("`Highscore"')
	\end{itemize}

\end{frame}





%--------------------------------------------
% Aufbau ------------------------------------
%--------------------------------------------
\begin{frame}\frametitle{Aufbau}
	\centering
	\includegraphics[width=0.99\textwidth]{aufbau_nah.jpg} \\
	{\small \textbf{Abbildung:} Nahansicht des Steckbrett-Aufbaus}
\end{frame}





%--------------------------------------------
% Ablauf ------------------------------------
%--------------------------------------------
\begin{frame}\frametitle{Ablauf}
	\movie[width=12cm,height=7.1cm,showcontrols]{\includegraphics[width=1.0\linewidth]{videobild.jpg}}{geschnitten.mp4}
\end{frame}



%--------------------------------------------
% Schwierigkeiten1 --------------------------
%--------------------------------------------

\begin{frame}[fragile]\frametitle{Schwierigkeiten}\framesubtitle{(Pseudo)Zufallsgenerator}
	\begin{itemize}\itemsep3ex
		\item Verfügbare Funktion \verb|random(min, max)| ist vollständig deterministisch
		\item Erzeugte Sequenz wäre bei jedem Start des Boards identisch
		\item Benötigen echtes Zufallselement
		\item \textbf{Lösung:} Verwende unbenutzten analogen Input-Pin für den Seed
		\begin{verbatim} randomSeed(analogRead(1)) \end{verbatim}

	
	\end{itemize}

\end{frame}


%--------------------------------------------
% Schwierigkeiten2 --------------------------
%--------------------------------------------

\begin{frame}[fragile]\frametitle{Schwierigkeiten}\framesubtitle{Speicherung der Sequenz}
	\begin{itemize}\itemsep3ex
		\item Prinzipiell unendlich lange Sequenz denkbar ("`Gedächtniskünstler"')
		\item (Statische) Arrays ungeeignet
		\item Aus C++ bekannter \verb|std::vector| liefert gewünschte Funktionalität, ist jedoch nicht verfügbar
		\item \textbf{Lösung:} Eigene, minimalistische Vector-Klasse mit gewünschten Funktionen
		\begin{itemize}\itemsep1ex
			\item Speichere Sequenz in Array
			\item Array voll: Erzeuge neues Array mit doppelter Größe und kopiere Elemente 
		\end{itemize}
	\end{itemize}

	
	
\end{frame}




%--------------------------------------------
% Ausblick ----------------------------------
%--------------------------------------------


\begin{frame}\frametitle{Ausblick}\framesubtitle{Zusätzliche Funktionen}
	\begin{itemize}\itemsep3ex
		\item Einstellung der Geschwindigkeit
		\item Mehrspieler-Modus
	\end{itemize}

\end{frame}


\end{document}
